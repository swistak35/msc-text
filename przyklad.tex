% Opcje klasy 'iithesis' opisane sa w komentarzach w pliku klasy. Za ich pomoca
% ustawia sie przede wszystkim jezyk i rodzaj (lic/inz/mgr) pracy, oraz czy na
% drugiej stronie pracy ma byc skladany wzor oswiadczenia o autorskim wykonaniu.
\documentclass[declaration,shortabstract]{iithesis}

\usepackage[utf8]{inputenc}

%%%%% DANE DO STRONY TYTUŁOWEJ
% Niezaleznie od jezyka pracy wybranego w opcjach klasy, tytul i streszczenie
% pracy nalezy podac zarowno w jezyku polskim, jak i angielskim.
% Pamietaj o madrym (zgodnym z logicznym rozbiorem zdania oraz estetyka) recznym
% zlamaniu wierszy w temacie pracy, zwlaszcza tego w jezyku pracy. Uzyj do tego
% polecenia \fmlinebreak.
\polishtitle    {Analiza statyczna języka Ruby}
\englishtitle   {Static analysis for Ruby language}
\polishabstract {\ldots}
\englishabstract{\ldots}
% w pracach wielu autorow nazwiska mozna oddzielic poleceniem \and
\author         {Rafał Łasocha}
% w przypadku kilku promotorow, lub koniecznosci podania ich afiliacji, linie
% w ponizszym poleceniu mozna zlamac poleceniem \fmlinebreak
\advisor        {prof. Witold Charatonik}
%\date          {}                     % Data zlozenia pracy
% Dane do oswiadczenia o autorskim wykonaniu
%\transcriptnum {}                     % Numer indeksu
%\advisorgen    {dr. Jana Kowalskiego} % Nazwisko promotora w dopelniaczu
%%%%%

%%%%% WLASNE DODATKOWE PAKIETY
%
%\usepackage{graphicx,listings,amsmath,amssymb,amsthm,amsfonts,tikz}
%
%%%%% WŁASNE DEFINICJE I POLECENIA
%
%\theoremstyle{definition} \newtheorem{definition}{Definition}[chapter]
%\theoremstyle{remark} \newtheorem{remark}[definition]{Observation}
%\theoremstyle{plain} \newtheorem{theorem}[definition]{Theorem}
%\theoremstyle{plain} \newtheorem{lemma}[definition]{Lemma}
%\renewcommand \qedsymbol {\ensuremath{\square}}
% ...
%%%%%

\begin{document}

\chapter{Wprowadzenie}

Współcześnie powstaje wiele narzędzi ułatwiających i przyśpieszających pracę programistów.
W szczególności dla wielu języków programowania istnieją zintegrowane środowiska programistyczne (IDE), które próbują dostarczyć jak najwięcej takich narzędzi w jednym spójnym środowisku. Jednymi z najbardziej podstawowych funkcjonalności takich środowisk są:

\begin{itemize}
\item autouzupełnianie pełnej nazwy aktualnie pisanej stałej, zmiennej lub nazwy funkcji
\item skok do definicji (stałej, funkcji lub zmiennej)
\item informacja o typie zmiennej
\item refactoring - zmiana nazwy klasy (globalnie, w całym projekcie), wydzielenie metody, zmiennej, etc.
\end{itemize}

Niniejsza praca przedstawia jak można wykorzystać analizę statyczną plików źródłowych, aby dostarczyć powyższe funkcjonalności w języku Ruby, który z racji swojej dynamicznej natury stawia pewne problemy mało istotne w językach statycznie typowanych.

\section{Założenia i cele}

\subsection{Utrudnienia w językach słabo typowanych}

Zintegrowane środowiska nie są niczym nowym i są popularne wśród programistów od wielu lat.
Jednakże, implementacja narzędzi bazujących na analizie kodu źródłowego zależy od niektórych cech języków programowania.
W językach silnie typowanych jest to znacznie prostsze, ponieważ gdy wiemy już jakiego typu jest zmienna - lista dostępnych nazw funkcji jest łatwa do obliczenia (autouzupełnianie).
Podobnie, języki silnie typowane często mają \textit{static dispatch}, więc w momencie kompilacji wiemy gdzie jest kod źródłowy, który zostanie użyty w danym wywołaniu funkcji (skok do definicji).
Oczywiście, w każdym języku możemy używać konstrukcji (np. w C, wywołanie funkcji która jest przechowana we wskaźniku), które to zadanie znacznie utrudniają, ale nie używa się ich zbyt często.
W językach dynamicznie typowanych (takich jak Ruby, Python), gdzie język nie dostarcza kompilatora, który dostarczałby informacji o kodzie źródłowym, informacje te musimy sobie znaleźć sami.

\subsection{Utrudnienia w Ruby}

W języku Ruby, o którym jest ta praca, statyczne wywnioskowanie definicji jest szczególnie ciężkie z dwóch powodów:
\begin{itemize}
\item wielu różnych dostępnych konstrukcji języka, takich jak wielokrotne dziedziczenie, klasy singletonowe, \textit{prepend} (przypis: nie wiem jak ten mechanizm się nazywa w nomenklaturze języków programowania), metaprogramowanie, które pozwalają na tworzenie skomplikowanych obiektów i hierarchii klas
\item społeczności rubiego, która w udostępnianych bibliotekach oraz projektach chętnie korzysta z tych skomplikowanych konstrukcji oraz metaprogramowania. Z jednej strony, duża ekspresja języka pozwala na szybkie wytwarzanie oprogramowania, z drugiej strony, analiza statyczna kodu, a nawet analiza kodu przez programistę jest trudniejsza (problemy podobne, z którymi zmagał się Smalltalk)
\end{itemize}

\subsection{Aktualny stan}

Do tworzenia oprogramowania w Ruby, programiści najczęściej wykorzystują Rubymine, Vim, Emacs, Atom (przypis: strzał na razie, ale nic istotnego to nie zmieni).
Rubymine jest jedynym IDE w tym rankingu, pozostałe narzędzia to tylko edytory tekstu, wspierane wtyczkami.
Rubymine dostarcza funkcjonalności wspomnianych we wstępie, jednak jest to narzędzie płatne i zamknięte.
Edytory tekstu same w sobie nie posiadają tych funkcjonalności, ale istnieją wtyczki, które w jakiejś części te funkcjonalności dostarczają. Jak to bywa z analizą statyczną, efektywność tych narzędzi to pewne spektrum, jedne narzędzia dostarczają lepsze autouzupełnianie, a inne gorsze, nie jest to binarne i jakość tych funkcjonalności ma duże znaczenie dla programisty.
Najpopularniejsze wtyczki to:
\begin{itemize}
\item CTags - zbiera symbole z całego projektu i ich lokalizacje, nie radzi sobie w żaden sposób z metaprogramowaniem i nie potrafi dostarczyć tych funkcjonalności zależnie od kontekstu (np. skok do definicji metody zależy od miejsca, w którym ją się wywołuje)
\item Robe - dostarcza informacje zależne od kontekstu, ale wykorzystuje bardzo proste heurystyki do zawężenia wyników
\item Solargraph - dostarcza podobne funkcjonalności do tych wymienionych we wstępie (oprócz \textit{refactoringu}), stosuje do tego analizę statyczną oraz wykorzystuje inne źródła (takie jak dokumentacja YARD)
\item RGL - odmiennie od pozostałych narzędzi, dostarcza system typów i inferencję typów do ruby. Wymaga jednak adnotacji klas i metod, tj. inferencja typów zachodzi tylko w ``otypowanych'' przez programistę metodach.
\end{itemize}

\subsection{Założenia i cele}

Celem pracy jest napisanie wtyczki działającej w zwykłych edytorach tekstu, która będzie oferować narzędzia wspierające pracę programisty.
Nie powinna wymagać od programisty zaawansowanej konfiguracji oraz żadnej pracy ze strony programisty (w szczególności pisania adnotacji typów).
Powinna dostarczyć jak najbardziej trafne informacje, głównie wykorzystując analizę statyczną, ale również inne źródła, jeśli jest to możliwe.
Informacje dostarczane przez wtyczkę niekoniecznie muszą być bezpieczne (w rozumieniu bezpiecznej analizy statycznej), bo jest to tylko wsparcie programisty, pozwalające sprawniej mu się poruszać po kodzie i go pisać.
Powinna brać pod uwagę praktyczne problemy wynikające nie tylko z technicznych zawiłości języka, ale również praktycznych.
W szczególności, powinna jak najlepiej sobie radzić z metaprogramowaniem, które jest używane przez społeczność.
Ponadto, powinna brać pod uwagę że kod źródłowy Ruby jest napisany w C (a więc, biblioteki standardowej nie możemy ``przeanalizować'') oraz w miarę możliwości fakt, że istnieją biblioteki korzystające z FFI (\textit{Foreign Function Interface}), a więc ich kod źródłowy jest napisany w innym języku (z reguły C).

\chapter{Rozwiązanie}

\section{Architektura}

Architektura wtyczki jest dwuczęściowa:
\begin{enumerate}
\item Protokół komunikacji z edytorem tekstu
\item Serwer, który dostarcza funkcjonalności, niezależny od edytora
\end{enumerate}

\subsection{Protokół komunikacji z edytorem tekstu}

We wtyczce wykorzystywany jest LSP (\textit{Language Server Protocol}).
Jest to protokół standaryzujący komunikację pomiędzy edytorami tekstu, a serwerami dostarczającymi informacji o projekcie.
Dzięki temu, chcąc napisać serwer, który potrafi komunikować się z $n$ edytorami tekstu, wystarczy zaimplementować w tym serwerze protokół LSP, który wspiera edytor tekstu (również często przez oddzielną wtyczkę). Ta część jest czysto inżynieryjna, więc w dalszej części pracy skupię się tylko na algorytmach wykorzystywanych w serwerze.

\section{Proces pozyskiwania informacji (?)}

\subsection{Funkcjonalności}

Przeanalizujmy jeszcze raz funkcjonalności, które chcielibyśmy uwzględnić w naszym serwerze. Pokażemy, że kluczowe jest:
\begin{enumerate}
\item Zebranie informacji o hierarchii klas, modułów oraz występujących w nich metod
\item Wywnioskowanie ``typu'' (klasy, czasami coś więcej) zmiennych
\end{enumerate}

\subsubsection{Informacja o typie zmiennej}

Informacja o typie sama przez się (przyp. kolokwializm?) jest informacją cenną dla programisty.
Implementacja wynika bezpośrednio z (2).
Mówimy o języku dynamicznie typowanym, więc w dalszej części pracy zostanie zdefiniowane czym jest typ, jednak ponieważ ruby jest językiem w pełni obiektowym i nie ma w nim typów prymitywnych, intuicyjnie typ możemy (z reguły, nie zawsze) powiązać z klasą obiektu, który jest przypisany do zmiennej.

\subsubsection{Skok do definicji}

Programista często dowiedzieć się jak wygląda definicja różnych bytów (?) w kodzie źródłowym:
\begin{enumerate}
\item definicja zmiennej lokalnej - nieskomplikowany przypadek i rzadko potrzebny, ponieważ krótkie metody z reguły powodują że definicja jest widoczna na pierwszy rzut oka
\item definicja stałej - wbrew pozorom, nie jest to proste, ponieważ w rubim stałe mogą być przypisywane ponownie (sic) oraz reguły zagnieżdżania i referencji nie są łatwe (z reguły zrozumiałe przez osoby z kilkuletnim doświadczeniem)
\item definicja metody - najtrudniejsza część, a jednocześnie najbardziej potrzebna. Będąc w kodzie źródłowym w miejscu wywołania metody, chcielibyśmy skoczyć do jej definicji. Metoda jest wywoływana na jakimś obiekcie, więc musimy znać ``typ'' zmiennej aby zlokalizować odpowiednią definicję.
\end{enumerate}

\subsubsection{Autouzupełnianie}

Podobnie jak wyżej, musimy znać typ zmiennej aby wiedzieć jakie metody powinniśmy zasugerować programiście.

\subsubsection{Refactoring}

Aby móc przeprowadzać w miarę bezpieczne automatyczne zmiany kodu źródłowego, takie jak:
\begin{itemize}
\item Zmiana nazwy klasy (globalnie, w całym projekcie)
\item Wydzielenie metody/zmiennej
\item \textit{Inline} (?) metody/zmiennej
\end{itemize}
również potrzebujemy informacji o typie zmiennej (przyp. rozwinąć dlaczego).

\subsection{Indeksowanie}

Uruchomienie serwera rozpoczyna przeindeksowanie wszystkich plików. Indeksowanie polega na:
\begin{enumerate}
\item Sparsowaniu pliku źródłowego do AST (\textit{abstract syntax tree})
\item Przejście przez AST funkcją dodającą wierzchołki i krawędzie do DFG (\textit{data flow graph}) oraz zbierającą informacje o definicjach klas, modułów i metod. Wierzchołki są często przypisane do danego węzła w AST, a więc również ścieżki i pozycji w pliku.
\item Uruchomienie funkcji wnioskującej typ wszystkich wierzchołków na podstawie informacji w grafie i hierarchii klas.
\end{enumerate}

\section{Budowanie grafu}

(na razie mniej więcej wymienione części o których trzeba powiedzieć)

\subsection{Wstęp}

Graf który budujemy jest skierowany i może zawierać cykle. Wierzchołki są etykietowane, tj. każdy wierzchołek ma swój ``rodzaj'' oraz być może dodatkowe parametry (zależnie od rodzaju). Ponadto wierzchołek może mieć przypisaną lokalizację w kodzie źródłowym. Lokalizacja składa się ze ścieżki do pliku, pozycji początkowej w kodzie źródłowym (numer linii i numer kolumny) oraz pozycji końcowej w kodzie źródłowym. Krawędzie są nieetykietowane.

\subsection{Literały}

Na początek warto popatrzeć na najprostsze AST, czyli literały. AST literału liczby całkowitej (np. \texttt{42}) zostanie przekształcony na pojedyńczy wierzchołek rodzaju \texttt{int}. Rodzaj będzie potrzebny funkcji wnioskującej typy, która na podstawie rodzaju, dodatkowych parametrów wierzchołka oraz wierzchołków z krawędzi wejściowych będzie decydowała jakiego typu jest dany wierzchołek. W tym prostym przypadku nie zwracamy uwagi na krawędzie wejściowe (bo ich, z metody budowania grafu, dla wierzchołka tego rodzaju nawet nie będzie) tylko przypisujemy wszystkim wierzchołkom rodzaju \texttt{int} typ \texttt{Nominal(Integer)}.

Tak samo jest z literałami: liczbami wymiernymi, zmiennopozycyjnymi, złożonymi oraz wartościami \texttt{true}, \texttt{false} oraz \texttt{nil}.

Podobnie sytuacja ma się w przypadku literałów tekstowych oraz symboli. Drobna różnica jest taka, że tekst może być z interpolacją (np. \texttt{``foo\#\{somevariable\}bar''}), a w takim przypadku najpierw rekurencyjnie przetwarzamy wszystkie interpolowane (?) wyrażenia, ale na końcu i tak przypisujemy wierzchołkowi rodzaju \texttt{str} zawsze typ \texttt{Nominal(String)}. Analogicznie jest w przypadku symboli.

\subsection{Literały tablic i słowników}

Pierwszym ciekawym przypadkiem jest literał tablicy, np. \texttt{[42, ``string'']}. Dla takiego AST, najpierw budujemy wierzchołki dla wszystkich poddrzew - w tym przykładzie dostaniemy dwa wierzchołki, odpowiednio rodzaju \texttt{int} oraz \texttt{str}. Następnie dodajemy wierzchołek rodzaju \texttt{array}, który będzie wynikiem przetwarzania tego AST, oraz dodamy krawędzie z wierzchołków \texttt{int} i \texttt{str} do wierzchołka \texttt{array}.

Funkcja wnioskująca typ dla wierzchołka rodzaju \texttt{array} zawsze patrzy na typy wszystkich wierzchołków krawędzi wejściowych, tworzy z nich jeden typ Union - w naszym przykładzie \texttt{Union(Integer or String)} oraz finalnie przypisuje temu wierzchołkowi typ \texttt{Array<Union(Integer or String)>}.


\subsection{Parę pozornie łatwych, a podłych przypadków}
\subsection{Klasy, moduły i inne stałe}
\subsection{Metody}
\subsection{Dziedziczenie}
\subsection{attr reader, private}
\subsection{Wywołania metod}
\subsection{Wywołania super}
\subsection{Klasy singletonowe}
\subsection{Include/extend/prepare}
\subsection{Zmienne globalne instancji, klasowe, instancji klas}
\subsection{Bloki / lambdy}
\subsection{Struktury}
\subsection{Biblioteka standardowa}

\section{Wnioskowanie typu}

W sporej części będą punkty jak wyżej, dodatkowo:

\subsection{Typy}
\subsection{Specjalny typ - Array/Tuple}
\subsection{Specjalny typ - Hash}
\subsection{Wywołania popularnych funkcji polimorficznych, Array.map}
\subsection{Szczególne funkcje w Object}

\section{Future work}

\subsection{More metaprogramming intelligence}
\subsection{Bundler, gems and libraries}
\subsection{YARD}
\subsection{Rust implementation}
\subsection{Caching}
\subsection{Tracking effects, ex. exceptions}
\subsection{Tracepoint}


%%%%% BIBLIOGRAFIA

%\begin{thebibliography}{1}
%\bibitem{example} \ldots
%\end{thebibliography}

\end{document}
