% Opcje klasy 'iithesis' opisane sa w komentarzach w pliku klasy. Za ich pomoca
% ustawia sie przede wszystkim jezyk i rodzaj (lic/inz/mgr) pracy, oraz czy na
% drugiej stronie pracy ma byc skladany wzor oswiadczenia o autorskim wykonaniu.
\documentclass[declaration,shortabstract]{iithesis}

\usepackage[utf8]{inputenc}

%%%%% DANE DO STRONY TYTUŁOWEJ
% Niezaleznie od jezyka pracy wybranego w opcjach klasy, tytul i streszczenie
% pracy nalezy podac zarowno w jezyku polskim, jak i angielskim.
% Pamietaj o madrym (zgodnym z logicznym rozbiorem zdania oraz estetyka) recznym
% zlamaniu wierszy w temacie pracy, zwlaszcza tego w jezyku pracy. Uzyj do tego
% polecenia \fmlinebreak.
\polishtitle    {Analiza statyczna języka Ruby}
\englishtitle   {English title}
\polishabstract {\ldots}
\englishabstract{\ldots}
% w pracach wielu autorow nazwiska mozna oddzielic poleceniem \and
\author         {Maksymilian Debeściak}
% w przypadku kilku promotorow, lub koniecznosci podania ich afiliacji, linie
% w ponizszym poleceniu mozna zlamac poleceniem \fmlinebreak
\advisor        {dr Jan Kowalski}
%\date          {}                     % Data zlozenia pracy
% Dane do oswiadczenia o autorskim wykonaniu
%\transcriptnum {}                     % Numer indeksu
%\advisorgen    {dr. Jana Kowalskiego} % Nazwisko promotora w dopelniaczu
%%%%%

%%%%% WLASNE DODATKOWE PAKIETY
%
%\usepackage{graphicx,listings,amsmath,amssymb,amsthm,amsfonts,tikz}
%
%%%%% WŁASNE DEFINICJE I POLECENIA
%
%\theoremstyle{definition} \newtheorem{definition}{Definition}[chapter]
%\theoremstyle{remark} \newtheorem{remark}[definition]{Observation}
%\theoremstyle{plain} \newtheorem{theorem}[definition]{Theorem}
%\theoremstyle{plain} \newtheorem{lemma}[definition]{Lemma}
%\renewcommand \qedsymbol {\ensuremath{\square}}
% ...
%%%%%

\begin{document}

%%%%% POCZĄTEK ZASADNICZEGO TEKSTU PRACY

\chapter{Wprowadzenie}

Współcześnie powstaje wiele narzędzi ułatwiających i przyśpieszających pracę programistów,
w szczególności dla wielu języków programowania istnieją zintegrowane środowiska programistyczne (IDE),
które próbują dostarczyć jak najwięcej takich narzędzi w jednym spójnym środowisku. Jednymi z najbardziej
podstawowych funkcjonalności takich środowisk są:
* autouzupełnianie pełnej nazwy aktualnie pisanej stałej, zmiennej lub nazwy funkcji
* skok do definicji (stałej, funkcji lub zmiennej)
* informacja o typie zmiennej
* refactoring - zmiana nazwy klasy (globalnie, w całym projekcie), wydzielenie metody, zmiennej, etc.

Zintegrowane środowiska nie są niczym nowym i są popularne wśród programistów od wielu lat.
Jednakże, implementacja powyższych funkcjonalności jest bardzo zależna od języka. W językach silnie typowanych jest to znacznie prostsze, ponieważ gdy wiemy już jakiego typu jest zmienna - lista dostępnych nazw funkcji jest łatwa do obliczenia (autouzupełnianie). Podobnie, języki silnie typowane często mają _static dispatch_, więc w momencie kompilacji wiemy gdzie jest kod źródłowy, który zostanie użyty w danym wywołaniu funkcji (skok do definicji). Oczywiście, możemy używać konstrukcji języka (np. w C, wywołanie funkcji która jest przechowana we wskaźniku), które to zadanie znacznie utrudniają, ale nie używa się ich zbyt często.

Zgoła odmienna sytuacja jest w językach dynamicznie typowanych (takich jak Ruby, Python), gdzie język nie dostarcza kompilatora, który dostarczałby informacji o kodzie źródłowym. W języku Ruby, o którym jest ta praca, statyczne wywnioskowanie definicji jest szczególnie ciężkie z dwóch powodów:
* wielu różnych dostępnych konstrukcji języka, takich jak wielokrotne dziedziczenie, klasy singletonowe, prepend, metaprogramowanie, które pozwalają na tworzenie skomplikowanych obiektów i hierarchii klas
* społeczności rubiego, która w udostepnianych bibliotekach chętnie korzysta z tych skomplikowanych konstrukcji oraz metaprogramowania

%%%%% BIBLIOGRAFIA

%\begin{thebibliography}{1}
%\bibitem{example} \ldots
%\end{thebibliography}

\end{document}
